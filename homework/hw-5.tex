\documentclass[12pt,a4]{article}

\usepackage{graphicx,subfigure,amsmath,amssymb,amsthm, boxedminipage,xcolor,float,enumerate,geometry}
\usepackage[lined,boxed]{algorithm2e}
\geometry{left=3cm,right=3cm,top=1.5cm,bottom=3.5cm}
\date{}

\title{
	Assignment Five\\
	\vspace{3mm}
	{\normalsize 516030910259 \textbf{Xinpeng Liu}}
}
\begin {document}
    \maketitle
	\paragraph{5.5} 
		b. and d. \\
		In shortest-job-first algorithm, when a process is executing, if there always appears a shorter process coming to queue, the algorithm will always choose the shorter one, which makes the original process suffer from starvation. \\
		In priority algorithm, if there is a process with the lowest priority, the chances are that it will suffer from starvation because every new process can make the algorithm suspend it and execute the new process.
	\paragraph{5.6} 
		\begin{enumerate}[a.]
			\item This process will be assigned twice the CPU time than the others.
			\item 
				Advantages: First, it won't suffer from starvation, every process can be executed. Second, it provides a better responsibility for processes.\\
				Disadvantages: First, the average waiting is usually longer. Second, the context switch can waste a lot of time. 
		\end{enumerate}
	\paragraph{5.9} 
		\begin{enumerate}[a.]
			\item In this algorithm, we can see if a process enters the ready queue later than another, its priority is always smaller. And the running process always has a higher priority. When a new process enters the queue, it has the lowest priority. These three features make the algorithm actually becomes a FCFS algorithm.
			\item In this algorithm, a later process has higher priority; the priority of a running process is higher than a waiting one; a new process has the highest priority. These features make the preemptive algorithm always execute the newest process. It becomes a last-come-first-served(LCFS) algorithm.
		\end{enumerate}
\end {document}
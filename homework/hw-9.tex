\documentclass[12pt,a4]{article}

\usepackage{graphicx,subfigure,amsmath,amssymb,amsthm, boxedminipage,xcolor,float,enumerate,geometry}
\usepackage[lined,boxed]{algorithm2e}
\geometry{left=3cm,right=3cm,top=1.5cm,bottom=3.5cm}
\date{}

\title{
	Assignment 9\\
	\vspace{3mm}
	{\normalsize 516030910259 \textbf{Xinpeng Liu}}
}
\begin {document}
	\maketitle
	\paragraph{9.4} The higher 20 bits of the address, which in this case is $11123_{16}$ will be transferred into a corresponding 6-bit physical page number by MMU. For example the corresponding 6-bit page number is $0F_{16}$. The lower 12 bits will be saved as page offset. Then these two parts will be spliced into an 18-bit physical address $0F456_{16}$. 
	\paragraph{9.8}.
	Consider the sequence `1,2,3,3,4,5,1` with a memory that can hold 4 pages. When fetching page 5, MFU will replace page 3, and LRU will replace page 1. Then for the next access, LRU will generate a page fault while MFU won't.\\
	For the opposite, consider the sequence `1,2,3,3,4,5,3` for the same memory. When fetching page 5, MFU replaces page 3, and LRU replaces page 1. Then for the next access, MFU will generate a page fault while LRU won't.
	\paragraph{9.13}
	\begin{enumerate}[a.]
		\item
			\begin{enumerate}
				\item 0
				\item When a new page is associated with the frame.
				\item When a page that associated with the frame is not needed.
				\item Choose the page with the least counter. Use FIFO if there are counters with the same value.
			\end{enumerate}
		\item 14 page faults.
		\item 11 page faults.
	\end{enumerate}
	\paragraph{9.14}
	0.8*1+0.18*2+0.02*20002=401.2us
\end {document}
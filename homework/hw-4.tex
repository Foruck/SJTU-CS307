\documentclass[12pt,a4]{article}

\usepackage{graphicx,subfigure,amsmath,amssymb,amsthm, boxedminipage,xcolor,float,enumerate,geometry}
\usepackage[lined,boxed]{algorithm2e}
\geometry{left=3cm,right=3cm,top=1.5cm,bottom=3.5cm}
\date{}

\title{
	Assignment Four \\
	\vspace{3mm}
	{\normalsize 516030910259 \textbf{Xinpeng Liu}}
}
\begin {document}
    \maketitle
	\paragraph{4.1} 
		\begin{enumerate}
			\item When we have a large number of operations, each consumes very little time. If we create a thread for each operation, the time used for creating and terminating threads is actually longer than the time used for operations. In this way, multi-threading doesn't work better.
			\item When we have a limited amount of resource, if we create lots of threads, to schedule those threads can take a long time. In this condition, single-threading is actually more efficient. 
		\end{enumerate}
	\paragraph{4.4} Global variables.
	\paragraph{4.5} Yes. When running on multiprocessor system, different threads can execute on different processors at the same. While on single-processor system, the processor has to switch frequently among threads, which is slower.
	\paragraph{4.8} 
		\begin{enumerate}[a.]
			\item Every thread is mapped to a processor, but the other processors are idle at the same time.
			\item Every thread is mapped to a processor, and there are no idle processors. But when a thread makes a blocking system call, the corresponding processor will be idle.
			\item Every processor is mapped to a thread, but the other threads are suspended at the same time. When a thread makes a blocking system call, it can be replaced by other threads, which increases the performance.
		\end{enumerate}
\end {document}
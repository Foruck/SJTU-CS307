\documentclass[12pt,a4]{article}

\usepackage{graphicx,subfigure,amsmath,amssymb,amsthm, boxedminipage,xcolor,float,enumerate,geometry}
\usepackage[lined,boxed]{algorithm2e}
\geometry{left=3cm,right=3cm,top=1.5cm,bottom=3.5cm}
\date{}

\title{
	Assignment 7\\
	\vspace{3mm}
	{\normalsize 516030910259 \textbf{Xinpeng Liu}}
}
\begin {document}
	\maketitle
	\paragraph{7.2}
	Mutual exclusion: A chopstick can only be held by one philosopher at one time.\\
	Hold and wait: When a philosopher holds a chopstick, it will keep waiting for the other.\\
	No preemption: The philosophers are equal. No one can grab chopsticks from others.\\
	Circular wait: $P_0$ waits for the chopstick held by $P_1$, $P_1$ waits for the chopstick held by $P_2$, ..., $P_4$ wait for the chopstick held by $P_1$.\\
	The condition of `No Preemption` can be avoided by giving the philosophers different priorities, and the philosopher with higher priority can grab chopstick from someone with lower priority
	\paragraph{7.3}.
	Containment doesn't need additional order definition, while circular-wait scheme needs.\\
	Under containment, there is actually only one thread can execute at one time, while multiple threads can run at the same time under circular-wait scheme.\\
	Circular-wait scheme can be more efficient than containment, but containment is easier.
	\paragraph{7.5}
	The changes of $b.$, $d.$ and $f.$ can be made safely whenever they happen.
\end {document}
\documentclass[12pt,a4]{article}

\usepackage{graphicx,subfigure,amsmath,amssymb,amsthm, boxedminipage,xcolor,float,enumerate,geometry}
\usepackage[lined,boxed]{algorithm2e}
\geometry{left=3cm,right=3cm,top=1.5cm,bottom=3.5cm}
\date{}

\title{
	Assignment 11\\
	\vspace{3mm}
	{\normalsize 516030910259 \textbf{Xinpeng Liu}}
}
\begin {document}
	\maketitle
	\paragraph{11.1} 
		\begin{enumerate}[a.]
			\item It's easy for the implementation of the allocation, but it may result in more internal fragmentation, 
			\item It's complicated for allocation, and there might be more external fragmentation.
			\item It achieves intermediate complexity and flexibility compared with the former two schemes.
		\end{enumerate}
	\paragraph{11.3}.
		\begin{enumerate}[a.]
			\item Yes. We can search the entire directory structure, find the empty spaces and link them into a new free-space list.
			\item 4.
			\item Store the pointer on the disk.
		\end{enumerate}
	\paragraph{11.6}
		\begin{enumerate}[a.]
			\item Assume $X$ is the start address of the file, $L$ is the logical address, $Y$ is the physical block number translated from logical address, and $Z$ is the block offset.
				\begin{enumerate}
					\item Contiguous. $Y$=$L/512$, $Z$=$L\%512$. Load the block with address $X+Y$ into memory. $Z$ is the offset.
					\item Linked. $Y$=$L/511$, $Z$=$L\%511$. Load the block with address $X$, then chase down the list by $Y+1$ elements. $Z$ is the offset.
					\item Indexed. $Y$=$L/512$, $Z$=$L\%512$. Load the block with address $X$, then the $Y$th line is the target physical block number. $Z$ is the offset.
				\end{enumerate}
			\item 
			\begin{enumerate}
					\item Contiguous. 1.
					\item Linked. 4.
					\item Indexed. 2.
				\end{enumerate}
		\end{enumerate}
\end {document}